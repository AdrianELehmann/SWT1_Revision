\documentclass[a4paper]{article}
\usepackage{etoc}
\usepackage[utf8]{inputenc}
\usepackage[T1]{fontenc}
\usepackage[light,condensed,math]{kurier}
\usepackage[ngerman]{babel}
\usepackage{ntheorem}
\usepackage{graphicx}
\usepackage{floatrow}
\usepackage{float}
\usepackage{hyperref}


\theoremstyle{break}
\newtheorem{defi}{Definition}[section]
\newtheorem{ex}{Beispiel}[section]
\newtheorem{why}{Vorteile}[section]
\newtheorem{whynot}{Nachteile}[section]
\title{SWT 1: Planung}
\author{Adrian E. Lehmann}

\begin{document}
	\maketitle
	\tableofcontents
	\newpage

	
\section{Anforderungserhebung}
	\textit{engl. Requirements elicitation}
\subsection{Methoden der Anforderungserhebung}	
\subsubsection{Fragebögen}
	\begin{why}
		\begin{enumerate}
			\item Kostengünstig
			\item Deckt viele Leute ab (gut für große Projekte)
			\item Relativ schnell
		\end{enumerate}
	\end{why}
	\begin{whynot}
		\begin{enumerate}
			\item Unpräzise
			\item Deckt evtl. nicht die eigentlichen Probleme ab
			\item Niedrige Motivation zum Ausfüllen unter Mitarbeitern
		\end{enumerate}
	\end{whynot}
\subsubsection{Interviews}
	\begin{why}
		\begin{enumerate}
			\item Sehr präzise Einsicht in Anwenderwünsche
		\end{enumerate}
	\end{why}
	\begin{whynot}
		\begin{enumerate}
			\item Deckt wenige Leute ab
			\item Teuer
			\end{enumerate}
	\end{whynot}
\subsubsection{Dokumenten- \& Aufgabenanalyse}
	\begin{why}
		\begin{enumerate}
			\item Direkte Einsicht in Abläufe
			\item Objektive Eindrücke
		\end{enumerate}
	\end{why}
	\begin{whynot}
		\begin{enumerate}
			\item Geht nicht auf Nutzerwünsche ein
			\item u. U. aufwendig
		\end{enumerate}
	\end{whynot}	
\subsubsection{Szenarien}
	\begin{why}
		\begin{enumerate}
			\item Direkte Einsicht in Abläufe
			\item Objektive Eindrücke
		\end{enumerate}
	\end{why}
	\begin{whynot}
		\begin{enumerate}
			\item Geht nicht auf Nutzerwünsche ein
			\item u. U. aufwendig
		\end{enumerate}
	\end{whynot}
\subsection{Anwendungsfälle}
//TODO
\subsection{Typen der Anforderungen}
\subsubsection{Funktionale Anforderungen}
\begin{defi}
	\begin{itemize}
		\item Beschreiben die Interaktionen zwischen dem System und der Systemumgebung, unabhängig von der Implementierung
		\item Beschreiben Benutzeraufgaben, die das System unterstützen muss
		\item Werden als Aktionen formuliert
	\end{itemize}
\end{defi}
\begin{ex}
	"Der Nutzer muss in der Lage sein Pizza zu bestellen"
\end{ex}
\subsubsection{Nicht-funktionale Anforderungen}
\begin{defi}
	\begin{enumerate}
		\item Aspekte, die nicht direkt mit dem funktionalen Verhalten des Systems in Verbindung stehen
		\item Beschreiben Eigenschaften des Systems oder der Domäne
		\item Werden als Einschränkungen oder Zusicherung (assertion)
		formuliert
	\end{enumerate}
\end{defi}
\begin{ex}
	"Der Nutzer soll seine Pizza binnen 2 Minuten bestellen können"
\end{ex}
\subsubsection{Einschränkungen}
\begin{defi}
	Durch den Kunden oder die Umgebung vorgegebene Aspekte:\newline
	\begin{enumerate}
		\item Implementierung
		\item Schnittstellen
		\item Einsatzumgebung
		\item Lieferumfang
		\item Rechtliches
			\subitem Lizenzen
			\subitem Zertifikate
			\subitem Datenschutz
	\end{enumerate}
\end{defi}
\begin{ex}
	"Die Implementierung muss in Brainfuck erfolgen" 
\end{ex}
\subsection{Validierung von Anforderungen}

\newpage
\section{Lastenheft}
\subsection{Gliederung}
\localtableofcontents
\subsubsection{Zielbestimmung}
Kurzfassung zu dem gewollten Produkt
\subsubsection{Produkteinsatz}
Beschreibung wer, wie und wo die Software verwendet wird (Zielgruppe, Plattform)
\subsubsection{Funktionale Anforderungen}
siehe oben
\subsubsection{Produktdaten}
Die Daten, die gehalten werden müssen
\subsubsection{Nichtfunktionale Anforderungen}
siehe oben
\subsubsection{Systemmodelle}
Anwendungsfälle und Akteure
\subsubsection{Glossar}
Definitionen, welche benötigt werden (Auf WiWi-Niveau - also quasi alles erklären)
\newpage
\section{Durchführbarkeitsuntersuchung}
\begin{enumerate}
	\item Fachliche Durchführbarkeit
		\subitem Softwaretechnische Realisierbarkeit
		\subitem Verfügbarkeit Entwicklungs- und Zielmaschinen
	\item Alternative Lösungsvorschläge
		\subitem Off the shelf software?
	\item Personelle Durchführbarkeit
	\item Prüfen der Risiken
	\item Ökonomische Durchführbarkeit
		\subitem Aufwands- und Terminschätzung
		\subitem Wirtschaftlichkeitsrechnung
	\item Rechtliche Gesichtspunkte
		\subitem Datenschutz
		\subitem Zertifizierung
		\subitem Relevante Standards
\end{enumerate}
\end{document}