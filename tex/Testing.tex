\documentclass[a4paper]{article}
\usepackage[utf8]{inputenc}
\usepackage[T1]{fontenc}
\usepackage[light,condensed,math]{kurier}
\usepackage[ngerman]{babel}
\usepackage{ntheorem}
\usepackage{graphicx}
\usepackage{floatrow}
\usepackage{float}
\usepackage{hyperref}

\theoremstyle{break}
\newtheorem{defi}{Definition}[section]
\newtheorem{ex}{Beispiel}[section]
\newtheorem{why}{Vorteile}[section]
\newtheorem{whynot}{Nachteile}[section]
\title{SWT 1: Testen}
\author{Sebastian Markgraf \& Adrian E. Lehmann}

\begin{document}
	\maketitle
	\begin{large}
	\begin{center} „Testing shows the presence of bugs, not their absence.“ - Edsger W. Dijkstra
	\end{center}
	\end{large}
	\tableofcontents
	\newpage
\section{Arten von Fehlern}
\section{Arten von Testhelfern}
\section{Fehlerklassen}
\section{Testverfahren}
\section{Testverfahren}
\subsection{Klassifikation}
\subsection{Kontrollflussorientierte (KFO) Testverfahren}
\subsection{Datenflussorientiert}
\subsection{Funktionale Tests}
\subsection{Leistungstests}
\subsection{Manuelle Prüfmethoden}
\subsection{Prüfprogramme}
\section{Testphasen}
\subsection{Komponententest}
\subsection{Integrationstest}
\subsection{Systemtest}
\subsection{Abnahmetest}
\section{Software-Inspektionen}
\subsection{Phasen}
\subsubsection{Vorbereitung}
\subsubsection{Individuelle Fehlersuche (IF)}
\subsubsection{Gruppensitzung (GS)}
\subsubsection{Nachbereitung}
\subsubsection{Prozessverbesserung}
\section{Integrationsstrategien}
\section{Systemtests}

\end{document}
