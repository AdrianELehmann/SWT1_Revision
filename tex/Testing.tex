\documentclass[a4paper]{article}
\usepackage[utf8]{inputenc}
\usepackage[T1]{fontenc}
\usepackage[light,condensed,math]{kurier}
\usepackage[ngerman]{babel}
\usepackage{ntheorem}
\usepackage{graphicx}
\usepackage{floatrow}
\usepackage{float}
\usepackage{hyperref}

\theoremstyle{break}
\newtheorem{defi}{Definition}[section]
\newtheorem{ex}{Beispiel}[section]
\newtheorem{why}{Vorteile}[section]
\newtheorem{whynot}{Nachteile}[section]
\title{SWT 1: Testen}
\author{Sebastian Markgraf \& Adrian E. Lehmann}

\begin{document}
	\maketitle
	\begin{large}
	\begin{center} „Testing shows the presence of bugs, not their absence.“ - Edsger W. Dijkstra
	\end{center}
	\end{large}
	\tableofcontents
	\newpage
        \section{Arten von Fehlern}
        \begin{defi}
          \textbf{Versagen} - Abweichung des Verhaltens der Software von der Spezifikation (ein Ereignis)
        \end {defi}
        \begin{defi}
          \textbf{Defekt} - ist ein Mangel in einem Softwareprodukt, der zu einem Versagen führen kann (ein Zustand)
        \end{defi}
        \begin{defi}
          \textbf{Irrtum} - ist eine menschliche Aktion, die einen Defekt verursacht. (ein Vorgang)
        \end{defi}
        \begin{defi}
          \textbf{Fehler} \textit{engl. Error} - Versagen | Defekt | Irrtum
        \end{defi}
        
        \section{Arten von Testhelfern}
        \begin{defi}
          \textbf{Stummel} \textit{engl. Stub} - ist ein rudimentär implementierter Teil der Software und dient als Platzhalter für noch nicht umgesetzte Funktionalität.
        \end{defi}
        \begin{defi}
          \textbf{Attrape} \textit{engl. Dummy} - simuliert die Implementierung zu Testzwecken
        \end{defi}
        \begin{defi}
          \textbf{Nachahmung} \texit{engl. Mock Object} - ist eine Attrappe mit zusätzlicher Funktionalität, wie bspw. das Einstellen der Reaktion der Nachahmung auf bestimmte Eingaben oder das Überprüfen des Verhaltens des 'Klienten'
        \end{defi}
        
        \section{Fehlerklassen}
        \begin{defi}
          \textbf{Anforderungsfehler} - Defekt im Pflichtenheft
        \end{defi}
        \begin {ex}
          \begin{enumerate}
          \item Inkorrekte Angabe der Benutzerwünsche
          \item Unvollständige Angaben über funktionale Anforderungen, Leistungsanforderungen...
          \item Inkosistenz verschiedener Anforderungen
          \item Undurchführbarkeit
            \end{enumerate}
        \end{ex}
        \begin{defi}
          \textbf{Entwurfsfehler} - Defekt in der Spezifikation
        \end{defi}
        \begin{ex}
          \begin{enumerate}
          \item Unvsollständige oder fehlerhafte Umsetzung der Anforderunge
          \item Inkosistenz der Spezifikation im Entwurf
          \item Inkosistenz zwischen Anforderung, Spezifikaion und Entwurf
          \end {enumerate}
          \end{ex}
        \begin {defi}
          \textbf{Implementierungsfehler} - Defekt im Programm
        \end{defi}
        \begin{ex}
          \begin{enumerate}
          \item Fehlerhafte Umsetzung der Spezifikation im Programm
          \end{enumerate}
        \end{ex}
        
        \section{Testverfahren}
\subsection{Klassifikation}
\subsection{Kontrollflussorientierte (KFO) Testverfahren}
\subsection{Datenflussorientiert}
\subsection{Funktionale Tests}
\subsection{Leistungstests}
\subsection{Manuelle Prüfmethoden}
\subsection{Prüfprogramme}
\section{Testphasen}
\subsection{Komponententest}
\subsection{Integrationstest}
\subsection{Systemtest}
\subsection{Abnahmetest}
\section{Software-Inspektionen}
\subsection{Phasen}
\subsubsection{Vorbereitung}
\subsubsection{Individuelle Fehlersuche (IF)}
\subsubsection{Gruppensitzung (GS)}
\subsubsection{Nachbereitung}
\subsubsection{Prozessverbesserung}
\section{Integrationsstrategien}
\section{Systemtests}

\end{document}
