\documentclass[a4paper]{article}
\usepackage[utf8]{inputenc}
\usepackage[T1]{fontenc}
\usepackage[light,condensed,math]{kurier}
\usepackage[ngerman]{babel}
\usepackage{ntheorem}
\usepackage{graphicx}
\usepackage{floatrow}
\usepackage{float}
\usepackage{hyperref}

\theoremstyle{break}
\newtheorem{defi}{Definition}[section]
\newtheorem{ex}{Beispiel}[section]
\newtheorem{why}{Vorteile}[section]
\newtheorem{whynot}{Nachteile}[section]
\title{SWT 1: Grundlagen}
\author{Sebastian Markgraf \& Adrian E. Lehmann}
\begin{document}
\maketitle
\section {Was ist Software}
\begin {defi}
  \textbf{Softwaretechnik} (engl. software engineering) ist die technologische \& organisatorische Disziplin zur systematischen Entwicklung und Pflege von Softwaresystemen, die spezifizierte funktionale und nicht-funktionale Attribute erfüllen.
\end {defi}

\begin {defi}
  \textbf{Softwareforschung} (engl. Software Engineering Research, Software Research) ist die Bereitstellung von Methoden, Verfahren und Werkzeugen für die Softwaretechnik
\end {defi}


\begin {defi}
  Das \textbf{Wasserfallmodell} ist ein lineares (nicht iteratives) Vorgehensmodell, das insbesondere für die Softwareentwicklung verwendet wird und das in aufeinander folgenden Projektphasen organisiert ist. Dabei gehen die Phasenergebnisse wie bei einem Wasserfall immer als bindende Vorgaben für die nächsttiefere Phase ein.
  Phasen:
  \begin {enumerate}
  \item Planung (\textit{engl. Systems Engineering})
  \item Definition (\textit{engl. Analysis})
  \item Entwurf (\textit{engl. Design})
  \item Implementierung, Modultests (\textit{engl. Coding})
  \item Testen (\textit{engl. Testing})
  \item Abnahme, Einsatz \& Wartung (\textit{engl. Maintenance})
  \end {enumerate}
  Dabei entstehen in jeder Phase Dokumente, welche für den nächsten Schritt verwendet werden.
  \begin {enumerate}
  \item Planung
	  \subitem Lastenheft, Projektkalkulation, Projektplan
  \item Definition
	  \subitem Pflichtenheft, Produktmodell, GUI-Modell, Benutzerhandbuch
  \item Entwurf
	  \subitem UML, Struktogramme
  \item Implementierung, Modultest
	  \subitem Code, Dokumentation, Modultests
  \item Testen
	  \subitem Testprotokoll
  \item Abnahme, Einsatz \& Wartung
	  \subitem Installiertes Produkt
	  \subitem Gesamtdokumentation
	  \subitem Abnahmeprotokoll
      \subitem Einführungsprotokoll
    \end {enumerate}
\end {defi}

\begin {why}
  \begin {enumerate}
    \item Klare Abgrenzung der Phasen
    \item Einfache Möglichkeit der Planung + Kontrolle
      \item Einfach und Verständlich
  \end {enumerate}
  
\end {why}
\begin {whynot}
  \begin {enumerate}
  \item Klare Abgrenzung sehr unrealistisch
  \item Rückschritte in Praxis oft unvermeidbar
  \item Unflexibel für Änderungen
  \item Fehler unter Umständen spät erkannt (Big Bang)
  \end {enumerate}
\end {whynot}


\end{document}
