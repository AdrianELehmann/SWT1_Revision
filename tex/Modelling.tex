\documentclass[a4paper]{article}
\usepackage[utf8]{inputenc}
\usepackage[T1]{fontenc}
\usepackage[light,condensed,math]{kurier}
\usepackage[ngerman]{babel}
\usepackage{ntheorem}
\usepackage{graphicx}
\usepackage{floatrow}
\usepackage{float}
\usepackage{hyperref}

\theoremstyle{break}
\newtheorem{defi}{Definition}[section]
\newtheorem{ann}{Bemerkung}[section]
\newtheorem{der}{Folgerung}[section]
\newtheorem{ex}{Beispiel}[section]
\newtheorem{why}{Vorteile}[section]
\newtheorem{whynot}{Nachteile}[section]
\title{SWT 1: Modellierung}
\author{M. Goetze}

\begin{document}
	\maketitle
	\tableofcontents
	\newpage
\section{Ziel}
\begin{itemize}
	\item Pflichtenheft
	\subitem Definition des zu erstellenden Systems (\textbf{vollständig}, \textbf{exakt})
\end{itemize}
\begin{ann}
	Beachte: Beschreibung \textbf{was} zu implementieren ist, nicht \textbf{wie} (z.B. Keine Algorithmen und Datenstrukturen)
\end{ann}

Ziel der Definitionsphase ist es ein Modell zu entwickeln. 

\begin{defi}[Modell]
	Abstraktion von existierenden oder imaginären
	\begin{itemize}
		\item Dingen, Personen, Konzepten etc.
		\item Abläufen
		\item Beziehungen
	\end{itemize}

\end{defi}

\begin{defi}[Funktionales Modell]
	\begin{itemize}
		\item Szenarien
		\item Anwendungsfall-Diagramme
		\item (aus dem Lastenheft)
	\end{itemize}
	
\end{defi} 
\begin{defi}[Objektmodell]
	\begin{itemize}
		\item Klassendiagramm
		\item Objektdiagramm (Instanzen im Klassendiagrammen)
		\item statisch
	\end{itemize}
\end{defi}
\begin{defi}[dynamisches Modell]
	\begin{itemize}
	\item Aktivitätsdiagramm
	\subitem Beschreibung paralleler / sequenzieller \textbf{Abläufe}
	\item Sequenzdiagramm
	\subitem Beschreibung der \textbf{Aufrufabläufe}
	\item Zustandsdiagramm
	\subitem \textbf{Zustandsübergänge} innerhalb eines einzelnen Objektes
\end{itemize}
\end{defi} 

\section{statisches Modell}
Entwicklung eines \textbf{statischen Modells}. 
\subsection{syntaktische Analyse}
Annäherung zum Erstellen eines UML-Klassendiagramms anhand eines Textdokumentes.
	
\begin{tabular}{|c|c|c|}
	\hline 
	\textbf{Wortart} & \textbf{Modellelement} & \textbf{Beispiel} \\
	\hline 
	Nomen & Klasse & Pizza, Spaghetto \\ 
	\hline 
	Namen & Exemplar & Walter \\ 
	\hline 
	Intransitives Verb & Botschaft & leben, schlafen  \\ 
	\hline 
	Transitives Verb & Assoziation & jmd. hassen, etw. zerstören \\ 
	\hline 
	Verb sein & Vererbung & ist eine (Art von) Störung  \\ 
	\hline 
	Verb haben & Aggregation & hat ein ...  \\ 
	\hline 
	Modalverb & Zusicherung  & müssen, sollen etc.  \\ 
	\hline 
	Adjektiv & Attribut & bizarr  \\ 
	\hline 
\end{tabular}
 
\section{Dynamisches Modell}
\begin{itemize}
	\item Quelle: Szenarien, Anwendungsfälle
	\item Ergebnis: Sequenz-, Aktivitätsdiagramm
	\item Zweck:
	\subitem  \textbf{Operationen} identifizieren
	\subitem Vollständigkeit und Korrektheit des statisches System \textbf{überprüfen}
	\subitem Grundlage für \textbf{Systemtests}
\end{itemize}
\section{Subsysteme}
\begin{itemize}
	\item \textbf{innerhalb} eines Subsystems \textbf{starke Kohäsion}
	\item \textbf{zwischen} den Subsystemen \textbf{schwache Kopplung}
\end{itemize}

\end{document}