\documentclass[a4paper]{article}
\usepackage[utf8]{inputenc}
\usepackage[T1]{fontenc}
\usepackage[light,condensed,math]{kurier}
\usepackage[ngerman]{babel}
\usepackage{ntheorem}
\usepackage{graphicx}
\usepackage{floatrow}
\usepackage{float}
\usepackage{hyperref}

\theoremstyle{break}
\newtheorem{defi}{Definition}[section]
\newtheorem{ex}{Beispiel}[section]
\newtheorem{why}{Vorteile}[section]
\newtheorem{whynot}{Nachteile}[section]
\title{SWT 1: Werkzeugkette}
\author{Sebastian Markgraf}
\begin{document}
    \maketitle
    \newpage

    \section{Versionskontrolle}
    \subsection{Softwarekonfigurationsverwaltung}
    \begin{defi}
      \textbf{Softwarekonfigurationsverwaltung} (engl. software configuration management) ist die Disziplin zur Verfolgung und Steuerung der Evolution von Software.
    \end{defi}

    \begin{ex}
      \begin {enumerate}
        \item RCS (1983)
        \item CVS (1990)
        \item SVN (2000)
        \item GIT (2005)
      \end {enumerate}
    \end{ex}

    \begin{flushleft}
      Besteht aus:
    \end{flushleft}
    \begin{enumerate}
    \item Quellelemente
    \item Bibliotheken
    \item Initialisierungsdaten
    \item Dokumentation
    \item Konfigurationsdateien für zu bauendes System  
    \end{enumerate}

    \subsection{Ausbuchen}
    \begin {defi}
      \textbf{Striktes Ausbuchen} erlaubt jeweils nur eine Ausbuchung mit Änderungsreservierung pro Version eines SEs
    \end {defi}

    \begin {defi}
      \textbf{Optimistisches Ausbuchen} macht keine Änderungsreservierung nötig. Demnach können mehrere Nutzer gleichzeitig eine Datei ändern.
    \end {defi}

    \newpage
    \subsection{GIT}
    \begin{enumerate}
    \item Alles ist ein Objekt
    \item Umgang mit Varianten (Branches)
    \item Kryptografische Sicherung der Historie
    \end {enumerate}

    \begin{flushleft}
      \textbf{GIT Commands}
    \end{flushleft}
    \begin{enumerate}
    \item help
    \item init
    \item clone
    \item add
    \item commit
    \item push
    \item fetch
    \item merge
    \item pull
    \item rebase \footnote{Die Gottheit: Verwende um nervige Merges aus der History zu bekommen.}
    \end{enumerate}

    \subsection {JUNIT}
		JUnit erlaubt einem das Unit Testen (für mehr Informationen siehe Dokument über "Testen")
     \subsection {Maven} 
     -- Nobody likes you --



    
    
\end{document}
