\documentclass[a4paper]{article}
\usepackage[utf8]{inputenc}
\usepackage[T1]{fontenc}
\usepackage[ngerman]{babel}
\usepackage{ntheorem}
\usepackage{graphicx}
\usepackage{floatrow}
\usepackage{float}
\usepackage{hyperref}

\theoremstyle{break}
\newtheorem{defi}{Definition}[section]
\newtheorem{ex}{Beispiel}[section]
\newtheorem{why}{Vorteile}[section]
\newtheorem{whynot}{Nachteile}[section]
\title{SWT 1: Werkzeugkette}
\author{Sebastian Markgraf}
\begin{document}
    \maketitle
    \newpage

    \section{Versionskontrolle}
    \subsection{Softwarekonfigurationsverwaltung}
    \begin{defi}
      Softwarekonfigurationsverwaltung (engl. software configuration management) ist die Disziplin zur Verfolgung und Steuerung der Evolution von Software.
    \end{defi}

    \begin{ex}
      \begin {enumerate}
        \item RCS (1983)
        \item CVS (1990)
        \item SVN (2000)
        \item GIT (2005)
      \end {enumerate}
    \end{ex}
    
     
\end{document}
